\documentclass[10pt]{beamer}

\usetheme[titleformat title=smallcaps, titleformat frame=allsmallcaps, titleformat section=smallcaps, block=fill, sectionpage=progressbar, subsectionpage=none, numbering=none, progressbar=foot, background = light]{metropolis}
\usecolortheme{default}
\usefonttheme{serif}

\usepackage[utf8]{inputenc}
\usepackage[T1]{fontenc}
\usepackage{multicol}
\usepackage{graphicx}
\usepackage{ragged2e}
\usepackage{animate}

%Informacje na stronie tytułowej:
\title{Od zapadki smoluchowskiego do silników brownowskich}
\subtitle{prowadzący: Bogdan Cichocki i Piotr Szymczak}
\author{Krzysztof Kwiatkowski \and Paweł Balawender}
\date{Warsztaty badawcze FUW 2018}
 
 
 
\begin{document}
	\frame{\titlepage}
	\begin{frame}
		\frametitle{Ruchy Browna - czym są?}
		\begin{minipage}{1\textwidth}
			\vspace{0.3cm}
			\begin{itemize}
				\item Występują na poziomie mezoskopowym ($10 nm - 10\mu m$).
				\item Obserwujemy je w zawiesinach.
				\item Są losowe i zygzakowate.
				\item Powstają na skutek fluktuacji częstości bombardowań cząstki.
				\item Ilość zderzeń na sekundę jest rzędu $10^{20}$, a fluktuacje $10^{10}$.
				\item Czas ruchu prostoliniowego jest rzędu $10^{-8}s$.
			\end{itemize}
		\end{minipage}
		\begin{minipage}{1\textwidth}
			\vspace{0.4cm}
			\Centering
			\animategraphics[autoplay,loop,height=4cm]{10}{brown_gif-}{0}{40}
		\end{minipage}
	\end{frame}
	\begin{frame}
		\frametitle{Trochę historii}
		\begin{minipage}{0.6\textwidth}
			\begin{itemize}
				\item 1827 - odkrycie przez Roberta Browna
				\item Jeden z największych problemów fizyki przez ponad 80 lat
				\item 1905 - publikacja Einsteina
				\item1906 - publikacja Mariana Smoluchowskiego
			\end{itemize}
		\end{minipage}
		\RaggedRight
		\begin{minipage}{0.38\textwidth}
			\Centering
			Marian Smoluchowski
			\vspace{0.4cm}
			\break
			\includegraphics[height=5cm]{smoluchowski.jpg}	
		\end{minipage}
	\end{frame}
	\begin{frame}
		\frametitle{Zapadka Smoluchowskiego}
		\begin{minipage}{1\textwidth}
			\justify
			\hspace{1em}
			Jest to mechanizm składający się z łopatek i mechanizmu zębatkowo-zapadkowego zaproponowany po raz pierwszy przez Smoluchowskiego. Urządzenie ma wykorzystywać fluktuacje bombardowania łopatek cząsteczekami ośrodka do wytworzenia pracy mechanicznej.
			\vspace{0.3cm}
		\end{minipage}
		\begin{minipage}{1\textwidth}
			\Centering
			\includegraphics[height=3.5cm]{silnik_brownowski_1.png}
		\end{minipage}
	\end{frame}
	\begin{frame}
		\frametitle{II zasada termodynamiki}
			\begin{minipage}{1\textwidth}
				\begin{itemize}
					\item Sformułowanie Kelvina: "Nie jest możliwy proces, którego jedynym skutkiem byłoby pobranie pewnej ilości ciepła ze zbiornika i zamiana go w równoważną ilość pracy."
					\item Oznacza to niemożliwość zbudowania perpetum mobile drugiego rodzaju.
				\end{itemize}
			\end{minipage}
			\begin{minipage}{1\textwidth}
				\Centering
				\vspace{0.4cm}
				\includegraphics[height=4cm]{2zasada.png}
			\end{minipage}
	\end{frame}
	\begin{frame}
		\frametitle{Dlaczego nie mogłaby działać}
			\begin{minipage}{1\textwidth}
				\begin{itemize}
					\item Łamie II zasadę termodynamiki.
					\item Zapadka także podlega fluktuacjom.
				\end{itemize}
		\end{minipage}
		\begin{minipage}{1\textwidth}
			\vspace{0.5cm}
			\Centering
			\includegraphics[height=3.5cm]{silnik_brownowski_1.png}
		\end{minipage}
		\vfill Co gdybyśmy spróbowali ograniczyć fluktuacje zapadki?
	\end{frame}	
	\begin{frame}
		\frametitle{Czy jednak da się coś zrobić?}
		\begin{minipage}{1\textwidth}
			\justify
			\hspace{1em}
			\begin{itemize}
				\item Załóżmy dwa układy o temperaturach $T_1<T_2$.
				\item Siła fluktuacji zależy od temperatury.
				\item Umieśćmy zapadkę w $T_1$, a łopatki w $T_2$.
				\item Fluktuacje działające na zapadkę są słabsze niż na łopatki.
			\end{itemize}
		\end{minipage}
		\begin{minipage}{1\textwidth}
			\vspace{0.4cm}
			\Centering
			\includegraphics[height=4cm]{silnik_brownowski_2.png}
		\end{minipage}
	\end{frame}
	\begin{frame}
		\frametitle{Opis matematyczny - ruch nadtłumiony}
		\begin{minipage}{1\textwidth}
			\Centering
			\includegraphics[height=4cm]{ruch.png}
		\end{minipage}
		\begin{minipage}{1\textwidth}
			\begin{itemize}
				\vspace{0.5cm}
				\item W tym ruchu siła działająca na cząstke $F=-\frac{dV}{dx}$ jest równoważona przez siłę oporu $-\gamma v$.
				\item Mamy więc związek $v=\frac{dx}{dt}=\frac{F}{\gamma}$.	
				\item Prędkość cząstki jest zależna od siły, która na nią działa.
			\end{itemize}
		\end{minipage}
	\end{frame}
	\begin{frame}
		\frametitle{Opis matematyczny - przesunięcie cząstki}
		\begin{minipage}{1\textwidth}
			\begin{equation}
				\nonumber
				\boxed{\Delta x=\frac F \gamma \Delta t+\alpha W(\Delta t)}
			\end{equation}
			\vspace{0.4cm}
			\begin{itemize}
				\item $F$ - stała siła działająca na cząstkę
				\item $\alpha = \sqrt{\frac{2 k_B T}{\gamma}}$ - natężenie szumu
				\item $W$ - składowa losowa ($<W>=0$, ale $<W^2>=\Delta t$)
			\end{itemize}
		\end{minipage}
	\end{frame}
	\begin{frame}
		\frametitle{Nasze symulacje - definicja układu}
		\begin{minipage}{1\textwidth}
			\Centering
			\includegraphics[height=4cm]{uklad.png}
		\end{minipage}
		\begin{minipage}{1\textwidth}
			\begin{itemize}
				\vspace{0.5cm}
				\item $a$ - parametr kontrolujący asymetrię potencjału
				\item $V_0$ - głębokość studni potencjału
				\item $L$ - długość komórki periodycznej
			\end{itemize}
		\end{minipage}
	\end{frame}
	\begin{frame}
		\frametitle{Rozkład Boltzmanna}
		\begin{minipage}{1\textwidth}
			\Centering
			 \begin{equation}
			\nonumber
			\boxed{ P(x) \sim e^{\frac{-V(x)}{k_B T}}}
			\end{equation}
		\end{minipage}
		\begin{minipage}{1\textwidth}
			\vspace{0.4cm}
			\Centering
			\includegraphics[height=6cm]{histogram.png}
		\end{minipage}
	\end{frame}
	\begin{frame}
		\frametitle{Ruch cząsteczki}
		\justify
		Z symulacji ruchu cząstki w układzie z asymetrycznym potencjałem i w zwykłym układzie bez potencjału uzyskamy kolejno dwa wykresy jej położenia:
		\break
		\begin{minipage}{0.47\textwidth}
			\vspace{0.2cm}
			\includegraphics[height=4cm]{plot1.png}
		\end{minipage}
		\begin{minipage}{0.47\textwidth}
			\vspace{0.2cm}
			\hspace{0.3cm}
			\includegraphics[height=4cm]{plot2.png}
		\end{minipage}
	\end{frame}
	\begin{frame}
		\frametitle{Przełączanie potencjału}
		\Centering
		\includegraphics[height=6cm]{switch.png}
	\end{frame}
	\begin{frame}
		\frametitle{Dlaczego tak się dzieje?}
			\Centering
			\includegraphics[height=5.5cm]{switch2.png}
			\vspace{0.3cm}
			\break
			źródło: H Linke, M T Downto, M J Zuckermann 2005, Performance characteristics of Brownian motors, dostęp 7 czerwca 2018, <https://www.researchgate.net/publication/7709238>
	\end{frame}
	\begin{frame}
		\frametitle{Zależność od częstości}
		\Centering
		\begin{equation}
		\nonumber
		\frac{2 k_B T}{x_1^2}<K<\frac{2 k_B T}{x_2^2}
		\hspace{3em}
		6,25 < K < 100
		\end{equation}
		\includegraphics[height=6cm]{switchsmooth.png}
	\end{frame}
	\begin{frame}
		\frametitle{Silniki brownowskie}
		\begin{minipage}{1\textwidth}
		To, co właśnie stworzyliśmy nazywawy silnikami brownowskimi. Podobne działają wewnątrz komórek w każdym z nas.
		\end{minipage}
		\begin{minipage}{1\textwidth}
		\vspace{0.2cm}
		\Centering
		\includegraphics[height = 5 cm]{brownianmotor.png}
		\vspace{0.2cm}
		\break
		źródło: R. Dean Astumian, Thermodynamics and Kinetics of a Brownian Motor, Science Vol. 276, 1997, dostęp 7 czerwca 2018, <http://science.sciencemag.org/content/276/5314/917>
		\end{minipage}
	\end{frame}
\end{document}
