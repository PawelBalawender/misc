\documentclass[11pt]{article}

\usepackage[T1]{fontenc}
\usepackage[polish]{babel}
\usepackage[utf8]{inputenc}
\usepackage{lmodern}
\usepackage{amsfonts}

\usepackage{graphicx}
\usepackage{calc}
\newlength{\depthofsumsign}
\setlength{\depthofsumsign}{\depthof{$\sum$}}
\newlength{\totalheightofsumsign}
\newlength{\heightanddepthofargument}

\newcommand{\nsum}[1][1.4]{
    \mathop{
        \raisebox
            {-#1\depthofsumsign+1\depthofsumsign}
            {\scalebox
                {#1}
                {$\displaystyle\sum$}
            }
    }
}

\selectlanguage{polish}

\title{Zadanie 1.}
\date{29 września 2018}
\author{Paweł Balawender}

\begin{document}

\maketitle

\section*{Problem}\
Rozstrzygnąć, czy istnieje taka dodatnia liczba całkowita $k$, że w zapisie dziesiętnym liczby $2^k$ każda z cyfr 0, 1,  \ldots, 9 występuje taką samą liczbę razy.

\section*{Rozwiązanie}
Zauważmy, że jeśli każda z cyfr danej liczby $x$ występuje w niej taką samą liczbę razy, to sumę $s \in\mathbb{N}$ tych cyfr zapisać możemy w postaci:
\begin{equation}
s=(1+2+3+4+5+6+7+8+9)n = 45n = 3 \cdot 15n
\end{equation}
gdzie $n\in\mathbb{N}$ jest liczbą dziewięciokrotnie mniejszą niż długość liczby\\
$x$ w zapisie dziesiętnym. Następnie:
\begin{equation}
s = 3 \cdot 15n \iff 3|s
\end{equation}
Na podstawie cechy podzielności przez 3 stwierdzamy, że:
\begin{equation}
3|s \iff 3|x \iff 3|2^k
\end{equation}\

co jest oczywistą sprzecznością, gdyż w wyniku rozkładu liczby $2^k$ na czynniki pierwsze otrzymamy iloczyn $k$ dwójek, nie będzie to więc liczba podzielna przez 3.\\
\\
Sprzeczność ta dowodzi, że nie istnieje taka liczba całkowita dodatnia $k$, że w zapisie dziesiętnym liczby $2^k$ każda z cyfr 0, 1, \ldots, 9 występuje taką samą liczbę razy.


\end{document}