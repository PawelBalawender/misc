\documentclass[12pt]{article}
\usepackage[left=2cm, right=5cm, top=2cm]{geometry}

\usepackage[T1]{fontenc}
\usepackage[polish]{babel}
\usepackage[utf8]{inputenc}
\usepackage{lmodern}
\usepackage{amsfonts}
\usepackage{siunitx}
\usepackage{textcomp}
\usepackage{ragged2e}

\selectlanguage{polish}


\title{Zadanie 2.}
\date{29 września 2018}
\author{Paweł Balawender}

\begin{document}
\maketitle

\section*{Problem}
\begin{justify}
Wysokości nierównoramiennego, ostrokątnego trójkąta $ABC$ przecinają się \linebreak
w punkcie $H$. Punkt $S$ jest środkiem tego łuku $BC$ okręgu opisanego na trójkącie
$BCH$, który zawiera punkt $H$. Wyznaczyć miarę kąta $BAC$, jeśli spełniona jest
równość $AH = AS$.

\section*{Rozwiązanie}

Kąt ten ma miarę \ang{60}.

\section*{Dowód}
B', C': spodki wysokości trójkąta ABC opuszczonych z wierzchołków B, C. \linebreak
Cechy kątów w trójkącie $B'CH$:
\begin{equation}\ang{90} + |B'CH| + |B'HC| = \ang{180}\end{equation}
\begin{equation} |B'HC| = \ang{180} - \ang{90} - |B'CH| \end{equation}
Punkty $A$, $B'$ i $C$ są współliniowe, podobnie jak $C$, $H$ i $C'$, więc:
\begin{equation}|B'CH| = |ACC'|\end{equation}
Cechy kątów w trójkącie $ACC'$:
\begin{equation}\ang{90} + |C'AC| + |ACC'|\end{equation}
\begin{equation}C'AC \equiv BAC = \ang{180} - \ang{90} - |ACC'|\end{equation}
Z równań (5) oraz (3) wynika:
\begin{equation}BAC \equiv B'HC\end{equation}
Kontynuując:
\begin{equation}|BAC| = |B'HC| = \ang{180} - |BHC|\end{equation}
Skoro kąty $BHC$ i $BSC$ są oparte na tym samym łuku:
\begin{equation}BHC \equiv BSC\end{equation}
Ponadto możemy zastosować wzór na miarę kąta opartego na łuku którego wierzchołek leży poza okręgiem:
\begin{equation}|BAC| = 1/2((\ang{360} - |BOC|) - |BOC|)\end{equation}
\begin{equation}|BAC| = \ang{180} - |BOC|\end{equation}
Korzystając z równości (7) i (9):
\begin{equation}BOC \equiv BHC\end{equation}
Korzystając z równości (8) i (11):
\begin{equation}BSC \equiv BOC\end{equation}

Odcinek $SO$ jest dwusieczną kątów $BOC$ oraz $BSC$, więc trójkąty $BSO$ oraz $CSO$ są równoramienne:
\begin{equation}|BS| = |BO| \land |CS| = |CO|\end{equation}
Ponadto, skoro punkty $B$, $S$ i $C$ leżą na jednym okręgu to:
\begin{equation}|OB| = |OS| \land |OC| = |OS|\end{equation}
Korzystając z równości (13) i (14):
\begin{equation}|OB| = |BS| = |SO| \land |CS| = |SO| = |CS|\end{equation}
Skoro trójkąty $BSO$ i $CSO$ są równoboczne i, jak wyżej wspomniane, $|SO|$ to dwusieczna $BSC$:
\begin{equation}|BSC| = 2 * \deg{60} = \deg{120}\end{equation}
Następnie korzystając z (10) i (12):
\begin{equation}|BAC| = \ang{180} - |BOC| = \ang{180} - |BSC|\end{equation}
I korzystając z (16) i (17):
\begin{equation}|BAC| = \ang{180} - \ang{120} = \ang{60}\end{equation}

\end{justify}
\end{document}