\documentclass[11pt]{article}

\usepackage[T1]{fontenc}
\usepackage[polish]{babel}
\usepackage[utf8]{inputenc}
\usepackage{lmodern}
\usepackage{amsfonts}

\usepackage{graphicx}
\usepackage{calc}
\newlength{\depthofsumsign}
\setlength{\depthofsumsign}{\depthof{$\sum$}}
\newlength{\totalheightofsumsign}
\newlength{\heightanddepthofargument}

\newcommand{\nsum}[1][1.4]{
    \mathop{
        \raisebox
            {-#1\depthofsumsign+1\depthofsumsign}
            {\scalebox
                {#1}
                {$\displaystyle\sum$}
            }
    }
}

\selectlanguage{polish}

\title{Zadanie 1.}
\date{29 września 2018}
\author{Paweł Balawender}

\begin{document}

\maketitle

\section*{Problem}
$P(x) = 0 \land Q(x) = 0 \land x \notin \mathbb{Q}$\\
$\iff P(x) = Q(x) \land Q(x) = 0 \land x \notin \mathbb{Q}$\\
$\iff x^3+ax^2+bx+c = x^3+bx^2+cx+a \land Q(x)=0 \land x \notin\mathbb{Q}$\\
$\iff a(x^2 - 1) + bx(1 - x) + c(1 - x) = 0 \land Q(x) = 0 \land x \notin \mathbb{Q}$\\
$\iff a(x+1)(x-1) - bx(x-1) - c(x-1) = 0$\\
$\iff (x-1)[a(x+1) -bx -c] = 0$\\
$\iff (x-1 = 0 \lor a(x+1)-bx-c = 0) $\\
$\iff (x = 1 \land Q(x) = 0 \land x \notin \mathbb{Q}) \lor (a(x+1)-bx-c=0 \land Q(x) = 0 \land x \notin \mathbb{Q})$\\
$\iff ax+a-bx-c=0 \land Q(x) = 0 \land x \notin \mathbb{Q}$\\
$\iff (a-b)x + a - c = 0 \land Q(x) = 0 \land x \notin \mathbb{Q}$\\

Teraz, skoro a, b i c są wymierne, to również a-b oraz a-c są wymierne. Iloczyn liczby wymiernej i niewymiernej jest liczbą niewymierną. Suma liczby wymiernej i niewymiernej jest liczbą niewymierną.
Zero jest wymierne, równość ta będzie więc spełniona tylko wtedy, gdy a-b = 0 = a-c, czyli b=c, co jest sprzeczne z treścią zadania.

Sprzeczność ta dowodzi, że nie istnieją takie trójki liczb a, b, c.


\end{document}