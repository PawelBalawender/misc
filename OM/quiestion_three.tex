\documentclass[11pt]{article}

\usepackage[left=2cm, right=5cm, top=2cm]{geometry}
\usepackage[T1]{fontenc}
\usepackage[polish]{babel}
\usepackage[utf8]{inputenc}
\usepackage{lmodern}
\usepackage{amsfonts}
\usepackage{ragged2e}
\usepackage{graphicx}
\usepackage{calc}
\newlength{\depthofsumsign}
\setlength{\depthofsumsign}{\depthof{$\sum$}}
\newlength{\totalheightofsumsign}
\newlength{\heightanddepthofargument}

\newcommand{\nsum}[1][1.4]{
    \mathop{
        \raisebox
            {-#1\depthofsumsign+1\depthofsumsign}
            {\scalebox
                {#1}
                {$\displaystyle\sum$}
            }
    }
}

\selectlanguage{polish}

\title{Zadanie 1.}
\date{29 września 2018}
\author{Paweł Balawender}

\begin{document}
\maketitle

\section*{Problem}
\begin{justify}
Rozstrzygnąć, czy istnieją takie parami różne liczby wymierne $a$, $b$, $c$, że wielomiany
$P(x) = x^3 + ax^2 + bx + c$\quad i\quad$Q(x) = x^3 + bx^2 + cx + a$
mają wspólny pierwiastek niewymierny.
\section*{Rozwiązanie}
Nie.
\section*{Dowód}
\begin{equation}P(x) = 0 \quad\land\quad Q(x) = 0 \land x \notin \mathbb{Q}\end{equation}
\begin{equation}P(x) = Q(x) \quad\land\quad Q(x) = 0 \land x \notin \mathbb{Q}\end{equation}
\begin{equation}x^3+ax^2+bx+c = x^3+bx^2+cx+a \quad\land\quad Q(x)=0 \land x \notin\mathbb{Q}\end{equation}
\begin{equation}a(x^2 - 1) + bx(1 - x) + c(1 - x) = 0 \quad\land\quad Q(x) = 0 \land x \notin \mathbb{Q}\end{equation}
\begin{equation}a(x+1)(x-1) - bx(x-1) - c(x-1) = 0 \quad\land\quad Q(x) = 0 \land x \notin \mathbb{Q}\end{equation}
\begin{equation}(x-1)(a(x+1) -bx -c) = 0 \quad\land\quad Q(x) = 0 \land x \notin \mathbb{Q}\end{equation}
\begin{equation}(x-1 = 0 \quad\lor\quad a(x+1)-bx-c = 0) \quad\land\quad Q(x) = 0 \land x \notin \mathbb{Q}\end{equation}
\begin{equation}(x = 1 \land Q(x) = 0 \land x \notin \mathbb{Q})\quad \lor\quad (a(x+1)-bx-c=0 \land Q(x) = 0 \land x \notin \mathbb{Q})\end{equation}
\begin{equation}ax+a-bx-c=0 \quad\land\quad Q(x) = 0 \land x \notin \mathbb{Q}\end{equation}
\begin{equation}(a-b)x + a - c = 0 \quad\land\quad Q(x) = 0 \land x \notin \mathbb{Q}\end{equation}
Skoro liczby $a$, $b$ i $c$ są wymierne, to zdanie to będzie prawdziwe tylko wtedy, gdy:
\begin{equation}a-b=0 \quad\land\quad a-c=0\end{equation}
Wtedy:
\begin{equation}a=b\end{equation}
Co jest sprzeczne z warunkami zadania - liczby muszą być parami różne.
\end{justify}
\end{document}