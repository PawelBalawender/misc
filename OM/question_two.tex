\documentclass[11pt]{article}

\usepackage[T1]{fontenc}
\usepackage[polish]{babel}
\usepackage[utf8]{inputenc}
\usepackage{lmodern}
\usepackage{amsfonts}
\usepackage{siunitx}
\usepackage{textcomp}

\usepackage{graphicx}
\usepackage{calc}
\newlength{\depthofsumsign}
\setlength{\depthofsumsign}{\depthof{$\sum$}}
\newlength{\totalheightofsumsign}
\newlength{\heightanddepthofargument}

\newcommand{\nsum}[1][1.4]{
    \mathop{
        \raisebox
            {-#1\depthofsumsign+1\depthofsumsign}
            {\scalebox
                {#1}
                {$\displaystyle\sum$}
            }
    }
}

\selectlanguage{polish}

\title{Zadanie 1.}
\date{29 września 2018}
\author{Paweł Balawender}

\begin{document}

\maketitle

\section*{Problem}\
Wysokości nierównoramiennego, ostrokątnego trójkąta $ABC$ przecinają się w punkcie $H$. Punkt $S$ jest środkiem tego łuku $BC$ okręgu opisanego na trójkącie $BCH$, który zawiera punkt $H$. Wyznaczyć miarę kąta $BAC$, jeśli spełniona jest równość $AH = AS$.

\section*{Rozwiązanie}
Przyjmijmy, bez straty ogólności, następującą modyfikację modelu:\\
punkt $H$ nie jest przecięciem się wysokości trójkąta $ABC$, a punktem przecięcia półprostych wychodzących z jego wierzchołków.
Oznaczmy przez $O$ środek okręgu opisanego na trójkącie $BHC$, przez $D$ punkt przecięcia odcinków $BC$ i $SO$, a przez $A'$, $B'$ i $C'$ punkty przecięcia półprostych wychodzących
odpowiednio z punktów $A$, $B$ i $C$, przechodzących przez punkt $H$.
W pierwotnym modelu półproste te byłyby wysokościami trójkąta, szukamy więc taki takiej wartości kąta $BAC$, dla której kąty $AA'B$, $BB'C$ i $CC'A$ są proste.

Zauważmy, że gdy $SD = DO$, to też $SO = BO = CO$ oraz odcinek $BC$ jest symetralną promienia $SO$, więc trójkąty $BSO$ i $OSC$ są równoboczne, a kąt $BSC=\ang{120}$.
Ponadto $\angle BHC = \angle BSC$, bo kąty te są oparte na tym samym łuku, a więc $BHC = \ang{120}$.
Jeśli $SD > DO$, $BHC < \ang{120}$, a jeśli $SD<DO$, $BHC>\ang{120}$.

Skoro $AH=AS$ oraz $OH=OS$, czworobok $AHOS$ jest deltoidem, a prosta wyznaczona przez punkty $HS$ dwusieczną kąta $\angle AHO \equiv \angle ASO$

Zauważmy, że dla $SD>DO$, kąt $\angle BB'C$ zawsze będzie mniejszy niż $\ang{90}$,\\
natomiast dla $SD<DO$, kąt $\angle BB'C$ zawsze jest większy niż $\ang{90}$.

Jedyną więc sytuacją, w której kąt BB'C ma miarę \ang{90}, jest sytuacja, w której $SD=DO$, czyli $BHC \equiv BOC = \ang{120}$.

Z własności kąta opartego na łuku, którego wierzchołek leży na zewnątrz okręgu wiemy, że $\angle BAC = 1/2((\ang{360} - \angle BOC) - \angle BOC) = 1/2(\ang{360} - 2 \angle BOC) = \ang{180} -\angle BOC$
Z czego wynika, że $\angle BAC = \ang{60}$.
\end{document}